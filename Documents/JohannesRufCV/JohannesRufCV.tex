\documentclass{resume}

\usepackage[left=0.75in,top=0.6in,right=0.75in,bottom=0.6in]{geometry} % Document margins
\usepackage[ngerman]{babel} 
\usepackage[utf8]{inputenc}

\name{Johannes Ruf}
\address{Blücherstr.24 \\ 76185 Karlsruhe \\ Deutschland }
\address{} 
\address{+49 178 180 6420 \\ johannes.m.ruf@gmail.com} 

\hyphenation{For-schungs-pro-jekt}


\begin{document}

%----------------------------------------------------------------------------------------
%	WORK EXPERIENCE SECTION
%----------------------------------------------------------------------------------------

\begin{rSection}{Relevante praktische Erfahrungen}

\begin{rSubsection}{European Institute for Energy Research (EIFER)}{seit Juni
2013}{Wissenschaftlicher Mitarbeiter}{Karlsruhe, Deutschland}
\item Grundlagenforschung im Bereich thermodynamischer Kreisprozesse in
Wärmekraftmaschinen
\item Angewandte Forschung im Bereich innovativer Kaltdampfprozesse 
\item Power-to-Heat Systeme an Fernwärmenetzen
\item Betreuung von mico-KWK Feldtestanlagen und Datenauswertung
\end{rSubsection}

%------------------------------------------------

\begin{rSubsection}{University of Alberta und KIT Institut für
Technische Chemie}{Juni 2013 - Feb. 2014}{Studentische
Hilfskraft und Diplomarbeit}{Edmonton, AB, Kanada und Karlsruhe, Deutschland}
\item Erzeugung von sekundären Pyrolysefeststoffen aus Biokoksen und
Pyrolyseölen (Primärpyrolyse)
\item Physikalische, chemische und optische Charakterisierung der sekundären
Feststoffe
\item Mitarbeit an einer wissenschaftlichen Veröffentlichung
\end{rSubsection}

%------------------------------------------------

\begin{rSubsection}{European Institute for Energy Research}{Juni -
Okt. 2010 und Nov. 2011 - Juli 2012}{Studentische
Hilfskraft}{Karlsruhe, Deutschland}
\item Mitarbeit in einem EU-Projekt zur Positionsbestimmung der
Brennstoffzellentechnologie  auf dem Markt für KWK-Systeme (FC EuroGrid)
\item Mitarbeit im Forschungsbereich Biomassenutzung und Heiztechnik für 
Kleinfeuerungsanlagen
\end{rSubsection}

%------------------------------------------------

\begin{rSubsection}{Bosch-Rexroth}{Mai - Aug. 2010}{Praktikum}{Lohr am Main,
Deutschland}
\item Mitarbeit in einem Projekt zur Automatisierung der Geometrievernetzung von
durchströmten Bauteilen für numerische Strömungssimulationen (MATLAB)
\item Aufnahme in das Bosch-Kontakt-Programm
\end{rSubsection}

%------------------------------------------------

\begin{rSubsection}{Iowa State University}{Nov. 2009 - Mai
2010}{Studienarbeit}{Ames, IA, USA}
\item Betreuung eines Feldtests an einer Pilotanlage Biomassevergasung
\item Messung der Input- und Outputgrößen und Auswertung der Messergebnisse
\item Spezialisierung auf die Ammoniakmessung und die Stickstoffumwandlungsprozesse
\end{rSubsection}

%------------------------------------------------

\begin{rSubsection}{Rollac Shutters of Texas}{Sept. - Dez.
2008}{Praktikum}{Pearland, TX, USA}
\item Aufgaben im Produktionsablauf der Komponentenherstellung
\item Mitarbeit am Entwurf neuer Produkte und dem Erstellen eines neuen Produktkataloges
\end{rSubsection}

%------------------------------------------------

\begin{rSubsection}{Otto Zimmermann GmbH}{Aug. - Sept.
2006}{Praktikum}{Saarbrücken, Deutschland}
\item Erlernen einfacher Metallverarbeitungsschritte
\item Entwicklung eines Gesamtverständnises der Fluidtechnik im Rahmen der
Arbeit eines Hydraulikanlagenbauers
\end{rSubsection}

\end{rSection}


%----------------------------------------------------------------------------------------
%	EDUCATION SECTION
%----------------------------------------------------------------------------------------

\begin{rSection}{Ausbildung}

{\bf Karlsruher Institut für Technologie in Karlsruhe, Deutschland} \hfill
{\em Sept. 2006 - April 2013}
\\
Maschinenbaustudium und  mit der Vertiefungsrichtung Energie- und Umwelttechnik 
\\
Fokus auf die Thematiken Energieumwandlung, Wasser und Siedlungsabfälle
\\
Abschluss des Vordiploms mit der Gesamtnote 1,9 (gut)
\\
Abschluss des Studiums mit dem Titel Dipl.-Ing. und der Note 1,4 (sehr gut) 


{\bf Universidad de Sevilla in Sevilla, Spanien} \hfill {\em Sept. 2010 -
Juli 2011}
\\
Teilnahme am ERASMUS-Programm: Fokus auf Chemieingenieurwesen und erneuerbare
Energien
\\
Erlernen der spanischen Sprache


{\bf Purdue University in West-Lafayette, IN, USA} \hfill {\em Jan. 2008 -
Mai 2009}
\\
Teilnahme am GEARE Austauschprogramm zwischen dem KIT und der Purdue University:
Mitarbeit an einem internationalen Konstruktionsprojekt mit chinesischen und 
amerikanischen Studenten. Absolvierung eines Auslandsemester und eines
Auslandspraktikums in den USA


{\bf Wirtschaftswissenschaftl. Gymnasium in Saarbrücken, Deutschland} \hfill
{\em Aug. 2002 - Juni 2005}
\\
Kenntnisse in VWL/BWL, Teilnahme an Kursen zur Ausbildung zum Mediator
\\
Abschluss des Abiturs in Mathematik, Wirtschaftslehre und Physik mit der
Gesamtnote 1,8 (gut)

\end{rSection}


%----------------------------------------------------------------------------------------
%	TECHNICAL STRENGTHS SECTION
%----------------------------------------------------------------------------------------

\begin{rSection}{Sonstige Informationen und Qualifikationen}

\begin{tabular}{ @{} >{\bfseries}l @{\hspace{6ex}} l }

Sprachkenntnisse	& Deutsch: Muttersprache (D)					\\
  					& Englisch: sehr gut in Wort und Schrift (C1)	\\
  					& Spanisch: gut (B2)							\\
  					& Französisch: gut (B1) 						\\
Programmiersprachen	& Python, MS-SQL, MATLAB, LaTex					\\
Software			& LabView, Eclipse, MS Office					\\
Interessen			& Ausdauersport, Schach, Reisen, Kennenlernen fremder Kulturkreise
\\
Geburtsdatum		& 27.01.1986 in Dudweiler, Deutschland			\\
Eltern				& Iris und Gerhard Ruf, Versicherungskauffrau und EDV-Kaufmann \\
\end{tabular}

\end{rSection}

%----------------------------------------------------------------------------------------
%	EXAMPLE SECTION
%----------------------------------------------------------------------------------------

%\begin{rSection}{Section Name}

%Section content\ldots

%\end{rSection}

%----------------------------------------------------------------------------------------

\end{document}

